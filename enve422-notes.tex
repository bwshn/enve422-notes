\documentclass[12pt]{article}
\usepackage[a4paper, right=1in, left=1in, top=1in,
bottom=1in, bindingoffset=0.2in]{geometry}
\usepackage{fontspec}
\setmainfont{Times New Roman}
\usepackage{microtype}
\usepackage[american]{babel}
\usepackage{parskip}
\usepackage[onehalfspacing]{setspace}
\usepackage{hyperref}
\usepackage{booktabs}


\title{{\Large ENVE 422}\\ {\small Treatment \& Disposal of Water \& Wastewater Sludge}}
\author{Bekir Şahin}
\date{2023}

\begin{document}
\maketitle
\tableofcontents
\pagebreak
\date{\begin{flushright}March 10, 2023\end{flushright}}
\section*{Syllabus}
\subsection{Introduction}

General premise: Provides general information on how general sludge is handled.


\emph{Cradle to cradle} approach to sludge:
How much sludge production, the characteristics of the sludge, how it can be made use of it beneficially.

\subsection{Grading}

\begin{itemize}
	\item[!] \textbf{No midterm for the course.} 
	\item Quiz, almost every other week. Maybe one bad quiz might be excluded\footnote{if we do have enough amount} 30\%
	\item Team, presentation, need a group, also have to contribute to the discussion!
	Presentation, 2 page summary, 1 question in the final exam. 25\%
	\item 3, 4 homework! 10\%
	\item Participation 5\%
	\item Final 35\%
\end{itemize}

Total = 105\%

Comment from the instructor: It is a hard course.

\date{\begin{flushright}March 17, 2023\end{flushright}}
\section{What is Sludge?}
How to use Sludge?
\begin{enumerate}
    \item Fertilizer
    Stabilization $\rightarrow$ reducing the microbial activity.
    \item Incineration $\rightarrow$ it has some calorific value, drying is a must.\\
    Brown coal, lignite (which is found in Turkey). Almost same in terms of calorific value.
\end{enumerate}
\subsection{Sludge Quantification}
Semi-solid material produced by water and wastewater treatment that itself needs further treatment for disposal into the environment.
\begin{itemize}
    \item High organic content
    \item Nutrients
    \item Pathogenic microorganisms
    \item Lots of water
\end{itemize}
Typical sludge generation\footnote{Calculated using census data and sludge production data}: 0.025-0.095 kg/capita/day\\
In Turkey: 0.035 kg/capita/day\\
Numbers depends on:
\begin{itemize}
    \item Wastewater characteristics
    \item Treatment types in the system
    \item Sludge treatment units
\end{itemize}
Although, the temperature, the climate, etc. are parameters, those three are the most affecting ones.\\
Sources:
\begin{itemize}
    \item Wastewater treatment:\begin{enumerate}
        \item Primary\\
        Primary treatment $\rightarrow$ For settleable solids with some intention of BOD\\
        Screens, disposed in landfills, no more treatment\\
        Primary treatment (called raw primary sludge)\begin{itemize}
            \item Odor
            \item Organics
            \item Water content\\
            Therefore, it should be treated.\\
            Typical treatment: anaerobic digesters, dewatering.\\
            Digesters do further treatment, then dewatering process comes along.\\
            Then, it can be used for beneficially.\\
            Anaerobic digesters do:\begin{enumerate}
                \item Organic content reduction
                \item Objectionable odor removal
                \item reduces the pathogens
            \end{enumerate}
        \end{itemize}
        \item Secondary\\
        Secondary treatment $\rightarrow$ removing BOD\\
        In advanced systems we don't have primary clarifier, final clarifier, returned activated sludge.\begin{enumerate}
            \item Return activated sludge: coming from final clarifier.
            \item Waste activated sludge: rest of the microbial flocs is removed from the system.\\
            To digester (with raw primary sludge *almost 50/50 to be stabilized*)
        \end{enumerate}
    \item Water treatment (Totally different)\\
    Water treatment $\rightarrow$ settling tank (major source)\\
    Not called digester, \textbf{Thickener}, then dewatering.\\
    Water treatment sludge (different from wastewater), affected by added coagulant and the quality of the water.\\
    Digestion  refers to the organic removing (for biomass).\\
    Thickening refers to squeezing the water.
    \item Industrial Sludge $\rightarrow$ hazardous waste (not included)
    \end{enumerate}
\end{itemize}

\begin{table}[h]
    \caption{The types of sludge, their physical concentrations and characteristics.}
    \label{tab:table_1}
    \centering
    \begin{tabular}{p{0.3\linewidth}rp{0.4\linewidth}}
        \toprule
        Sludge & Conc. (\%) & Characteristics \\
        \midrule
        Raw primary & 4-8\footnote{It is a huge number, very concentrated.} & not drained well on drying beds \newline dewatered mechanically\\
        Anaerobic primary digested & 6-10 & gas production, well on drying beds\\
        Wasted activated sludge & 0.5-1.5 & no odor, bio. active\\
        Mixed digested & 2-4 & produces gas, not as easy to dewater\\
        Aerobic digested & 1-3 & bio. active, not as easy to dewater\\
        Waste alum & 0.5-1.5 & not active, inorganic content\\
        \bottomrule
    \end{tabular}
\end{table}

Nitrogen, ammonia, nitrate, phosphorus, potassium $\rightarrow$ important for fertilizer.\\
Sodium, calcium iron are some inorganic content in the sludge.\\
Microorganisms: Viruses, fecal coliforms, salmonella...\\
Before and after $\rightarrow$ they are still there after the treatment.

\textbf{Quantities\footnote{A question in homework and quiz}:}
\[S_0 \rightarrow incoming\,BOD\,(mass/time)\]
\[X_0 \rightarrow incoming\,solids\,(mass/time)\]
Those are loads.\\
From primary, kX\textsubscript{0} to digester / up to 0.6 for k, good number if close to this.\\
hS\textsubscript{0} and (1-k)X\textsubscript{0} / h is 0.7 min, means 30\% removed with primary clarifier, since we are not removing BOD in the primary clarifier, it is expected.\\
After final clarifier:\\
ihS\textsubscript{0} / so i has to be really small (0.1, means 90\% removal) and X\textsubscript{f} after the removal.\\
Incoming to digesters:\\
1st: kX\textsubscript{0} came, as primary only jkX\textsubscript{0}, j = 0.8 means 20\% removal achieved for primary only.\\
2nd: (1-k)X\textsubscript{0}-X\textsubscript{f} + $\Delta$X\\
$\Delta$X is net solids produced by biological action.\\
Y (Yield) $\Delta$X/$\Delta$S\\
$\Delta$S = hS\textsubscript{0}-ihS\textsubscript{0}\\
Y = 0.5 for activated sludge;\\
Y = 0.2 for trickling filters, because it is biofilm.
\subsection{Sludge Characteristics}
Caused due to differences in the type and quality of wastewater and differences in the wastewater treatment processes and operations.\\
General sludge characteristics by typical parameters can be explained (even though it is time dependent).\\
\textbf{Physical characteristics:}
\begin{enumerate}
    \item Specific gravity\\
    Def: Ratio of  weight of the material to the weight of an equal volume of water.\\
    1L sludge weighing 1010g / 1000g water = 1.01 is the specific gravity.\\
    Most sludge have specific gravity of 1.0, almost equal to the weight of water, what might be the problem?\\
    \textbf{Shows similar behavior as water with gravity, physical separation is hard.}\\
    Sludge is suspension, solid and liquid and more than one solid (volatile and fixed components exist).
    \[\frac{1}{S}= \sum_{i=1}^{n} \frac{W_i}{S_i}\]
    \item Sludge density\\
    Again, very close to 1 (1.0... in examples).
    \item Solid concentration\\ % tree make for figure 5
    mg/L or \% solids\\
    if we assume the specific gravity of the slurry is 1.0,\\
    10,000 mg/L = 1\%\\
    For sludge with higher S values, the relationship will change.
    \item Settling characteristics\\
    How well it settles.
    \begin{itemize}
        \item Zone Settling Velocity\footnote{Hindered settling test} (ZSV)\\
        ZSV is in hyperbolically inverse relationship with concentration.
        \[ZSV = k * \frac{1}{C}\]
        Why is there such a relationship?\\
        \textbf{Higher concentration compresses in more time, the already settled particles creates interference.}\\
        Why do we care settling in second?
        \begin{enumerate}
            \item if not good settling, the return sludge will be low.\\
            return sludge $\rightarrow$ maintaining a good solid content, good MLSS results in good removal.
            \item if not settle, we lose them to effluent, all the work we did is gonna be wasted.
        \end{enumerate}
        \item Sludge Volume Index (SVI)\\
        Quick test.\\
        1L cylinder for 30 min.
        \[SVI = V_{30} * \frac{1000}{MLSS}\]
        An operational tool designed for the  treatment, more about settleability to gain an idea but not a good research too.\\
        SVI < 120 $\rightarrow$ well settling.\\
        120 < SVI < 150 $\rightarrow$ bulking occurs.\\
        150 < SVI $\rightarrow$ severely bulking, requires action.
        \item Particle size\\
        Micro level, cannot be defined by a single size / a distribution must be mentioned.
        Many dynamic affect the distribution.\\
        Viscosity, settling, dewatering are affected.\\
        Measurement is hard.
        \begin{enumerate}
            \item Filtration, thorough a series of different sized filters.
            \item Photographic techniques.
            \item Scattering laser light.\\
            Even you measure, dewatering settling rheology is still complicated.
        \end{enumerate}
        \item Floc structure and porosity\\
        Flocs formed from three major components:
        \begin{enumerate}
            \item microorganisms
            \item extracellular polymers
            \item water
        \end{enumerate}
    \end{itemize}
\end{enumerate}
Extracellular polymers (EPS) hold individual microorganisms $\rightarrow$ flocs causes separation an settling.\\
Filamentous organisms give strength to the structure.\\
EPS is a major component, creates a structure with water channels, gaps, reservoirs.\\
It has pores: porosity.\\
Activated sludge: 0.999


\end{document}
