\documentclass[12pt]{article}
\usepackage[a4paper, right=2cm, left=2cm, top=2cm,
bottom=2cm, bindingoffset=0.5cm]{geometry}
\usepackage[american]{babel}
\usepackage{parskip}
\usepackage{hyperref}
\usepackage{booktabs}
\usepackage{graphicx}
\usepackage{fontspec}
\setmainfont{Times New Roman}
\usepackage[onehalfspacing]{setspace}

\title{{\Large ENVE 422}\\ {\small Treatment \& Disposal of Water \& Wastewater Sludge}}
\author{Bekir Şahin}
\date{2023}

\begin{document}
\maketitle
\tableofcontents
\pagebreak
\date{\begin{flushright}March 10, 2023\end{flushright}}
\section*{Syllabus}
\subsection*{Introduction}
General premise: Provides general information on how general sludge is handled.\\
\textit{Cradle to cradle} approach to sludge: How much sludge production, the characteristics of the sludge, how it can be made use of it beneficially...
\subsection*{Grading}
\begin{itemize}
	\item[!] \textbf{No midterm for the course.} 
	\item Quiz, almost every other week. Maybe one bad quiz might be excluded\footnote{if we do have enough amount} \textbf{30\%}
	\item Team presentation, also have to contribute to the discussion!
	Presentation, 2 page summary, 1 question in the final exam. \textbf{25\%}
	\item 3, 4 homework! \textbf{10\%}
	\item Participation \textbf{5\%}
	\item Final \textbf{35\%}
\end{itemize}
Total = \textbf{105\%}\\
Comment from the instructor: It is a hard course.
\date{\begin{flushright}March 17, 2023\end{flushright}}
\section{What is Sludge?}
How to use Sludge?
\begin{enumerate}
    \item Fertilizer
    Stabilization $\rightarrow$ reducing the microbial activity.
    \item Incineration $\rightarrow$ it has some calorific value, drying is a must.\\
    Brown coal, lignite (which is found in Turkey). Almost same in terms of calorific value.
\end{enumerate}
\subsection{Sludge Quantification}
Semi-solid material produced by water and wastewater treatment that itself needs further treatment for disposal into the environment.
\begin{itemize}
    \item High organic content
    \item Nutrients
    \item Pathogenic microorganisms
    \item Lots of water
\end{itemize}
Typical sludge generation\footnote{Calculated using census data and sludge production data}: 0.025-0.095 kg/capita/day\\
In Turkey: 0.035 kg/capita/day\\
Numbers depends on:
\begin{itemize}
    \item Wastewater characteristics
    \item Treatment types in the system
    \item Sludge treatment units
\end{itemize}
Although, the temperature, the climate, etc. are parameters, those three are the most affecting ones.\\
Sources:
\begin{itemize}
    \item Wastewater treatment:\begin{enumerate}
        \item Primary\\
        Primary treatment $\rightarrow$ For settleable solids with some intention of BOD\\
        Screens, disposed in landfills, no more treatment\\
        Primary treatment (called raw primary sludge)\begin{itemize}
            \item Odor
            \item Organics
            \item Water content\\
            Therefore, it should be treated.\\
            Typical treatment: anaerobic digesters, dewatering.\\
            Digesters do further treatment, then dewatering process comes along.\\
            Then, it can be used for beneficially.\\
            Anaerobic digesters do:\begin{enumerate}
                \item Organic content reduction
                \item Objectionable odor removal
                \item reduces the pathogens
            \end{enumerate}
        \end{itemize}
        \item Secondary\\
        Secondary treatment $\rightarrow$ removing BOD\\
        In advanced systems we don't have primary clarifier, final clarifier, returned activated sludge.\begin{enumerate}
            \item Return activated sludge: coming from final clarifier.
            \item Waste activated sludge: rest of the microbial flocs is removed from the system.\\
            To digester (with raw primary sludge *almost 50/50 to be stabilized*)
        \end{enumerate}
    \item Water treatment (Totally different)\\
    Water treatment $\rightarrow$ settling tank (major source)\\
    Not called digester, \textbf{Thickener}, then dewatering.\\
    Water treatment sludge (different from wastewater), affected by added coagulant and the quality of the water.\\
    Digestion  refers to the organic removing (for biomass).\\
    Thickening refers to squeezing the water.
    \item Industrial Sludge $\rightarrow$ hazardous waste (not included)
    \end{enumerate}
\end{itemize}

\begin{table}[h]
    \caption{The types of sludge, their physical concentrations and characteristics.}
    \label{tab:table_1}
    \centering
    \begin{tabular}{p{0.3\linewidth}rp{0.4\linewidth}}
        \toprule
        Sludge & Conc. (\%) & Characteristics \\
        \midrule
        Raw primary & 4-8\footnote{It is a huge number, very concentrated.} & not drained well on drying beds \newline dewatered mechanically\\
        Anaerobic primary digested & 6-10 & gas production, well on drying beds\\
        Wasted activated sludge & 0.5-1.5 & no odor, bio. active\\
        Mixed digested & 2-4 & produces gas, not as easy to dewater\\
        Aerobic digested & 1-3 & bio. active, not as easy to dewater\\
        Waste alum & 0.5-1.5 & not active, inorganic content\\
        \bottomrule
    \end{tabular}
\end{table}

Nitrogen, ammonia, nitrate, phosphorus, potassium $\rightarrow$ important for fertilizer.\\
Sodium, calcium iron are some inorganic content in the sludge.\\
Microorganisms: Viruses, fecal coliforms, salmonella...\\
Before and after $\rightarrow$ they are still there after the treatment.
\begin{figure}[ht]
    \centering
    \includegraphics[width=0.95\textwidth]{SludgeQuantities.png}
    \caption{Flow chart of a sludge generation in a conventional wastewater treatment plant.}
    \label{fig:sludge}
\end{figure}
\[S_0 \rightarrow incoming\,BOD\,(mass/time)\]
\[X_0 \rightarrow incoming\,solids\,(mass/time)\]
Those are loads.\\
From primary, kX\textsubscript{0} to digester / up to 0.6 for k, good number if close to this.\\
hS\textsubscript{0} and (1-k)X\textsubscript{0} / h is 0.7 min, means 30\% removed with primary clarifier, since we are not removing BOD in the primary clarifier, it is expected.\\
After final clarifier:\\
ihS\textsubscript{0} / so i has to be really small (0.1, means 90\% removal) and X\textsubscript{f} after the removal.\\
Incoming to digesters:\\
1st: kX\textsubscript{0} came, as primary only jkX\textsubscript{0}, j = 0.8 means 20\% removal achieved for primary only.\\
2nd: (1-k)X\textsubscript{0}-X\textsubscript{f} + $\Delta$X\\
$\Delta$X is net solids produced by biological action.\\
Y (Yield) $\Delta$X/$\Delta$S\\
$\Delta$S = hS\textsubscript{0}-ihS\textsubscript{0}\\
Y = 0.5 for activated sludge;\\
Y = 0.2 for trickling filters, because it is bio-film.
\subsection{Sludge Characteristics}
Caused due to differences in the type and quality of wastewater and differences in the wastewater treatment processes and operations.\\
General sludge characteristics by typical parameters can be explained (even though it is time dependent).\\
\textbf{Physical characteristics:}
\begin{enumerate}
    \item Specific gravity\\
    Def: Ratio of  weight of the material to the weight of an equal volume of water.\\
    1L sludge weighing 1010g / 1000g water = 1.01 is the specific gravity.\\
    Most sludge have specific gravity of 1.0, almost equal to the weight of water, what might be the problem?\\
    \textbf{Shows similar behavior as water with gravity, physical separation is hard.}\\
    Sludge is suspension, solid and liquid and more than one solid (volatile and fixed components exist).
    \[\frac{1}{S}= \sum_{i=1}^{n} \frac{W_i}{S_i}\]
    \item Sludge density\\
    Again, very close to 1 (1.0... in examples).
    \item Solid concentration\\
    mg/L or \% solids\\
    if we assume the specific gravity of the slurry is 1.0,\\
    10,000 mg/L = 1\%\\ % explain the measurement
    from percent:\\
    For sludge with higher S values, the relationship will change.
    \end{enumerate}
\textbf{Settling characteristics:}\\
    How well it settles.
    \begin{itemize}
        \item Zone Settling Velocity\footnote{Hindered settling test} (ZSV)\\
        ZSV is in hyperbolically inverse relationship with concentration.
        \[ZSV = k * \frac{1}{C}\]
        Why is there such a relationship?\\
        \textbf{Higher concentration compresses in more time, the already settled particles creates interference.}\\
        Why do we care settling in second?
        \begin{enumerate}
            \item If not good settling, the return sludge will be low.\\
            return sludge $\rightarrow$ maintaining a good solid content, good MLSS results in good removal.
            \item If not settle, we lose them to effluent, all the work we did is gonna be wasted.
        \end{enumerate}
        \item Sludge Volume Index (SVI)\\
        Quick test.\\
        1L cylinder for 30 min.
        \[SVI = V_{30} * \frac{1000}{MLSS}\]
        An operational tool designed for the  treatment, more about settleability to gain an idea but not a good research too.\\
        SVI < 120 $\rightarrow$ well settling.\\
        120 < SVI < 150 $\rightarrow$ bulking occurs.\\
        150 < SVI $\rightarrow$ severely bulking, requires action.
        \item Particle size\\
        Micro level, cannot be defined by a single size / a distribution must be mentioned.
        Many dynamic affect the distribution.\\
        Viscosity, settling, dewatering are affected.\\
        Measurement is hard.
        \begin{enumerate}
            \item Filtration, thorough a series of different sized filters.
            \item Photographic techniques.
            \item Scattering laser light.\\
            Even you measure, dewatering settling rheology is still complicated.
        \end{enumerate}
        \item Floc structure and porosity\\
        Flocs formed from three major components:
        \begin{enumerate}
            \item microorganisms
            \item extracellular polymers
            \item water
        \end{enumerate}
    \end{itemize}
Extracellular polymers (EPS) hold individual microorganisms $\rightarrow$ flocs causes separation an settling.\\
Filamentous organisms give strength to the structure.\\
EPS is a major component, creates a structure with water channels, gaps, reservoirs.\\
It has pores: porosity.\\
Activated sludge porosity: 0.999
\date{\begin{flushright}March 24, 2023\end{flushright}}
\textbf{Distribution of Water in Sludge}\\
Sludge is two-phase slurry.\\
To carry the sludge, we must minimize the volume of the sludge.
Form of water in sludge determines the effectiveness of sludge treatment.\\
Besides drying, freezing is a good option for dewatering.
Freezing does not happen at 0 or lower like -3, it happened -20 degree Celsius.
Constituents of sludge:
Free (bulk) water (75\%):\\
Water not associated with solids and not influenced by solids.
Interstitial water:\\
Water trapped in the interstitial spaces of the flocs and organisms. this part can be removed by mechanical dewatering.
Vicinal water:\\
Multiple layers of water molecules help tightly to the particle surface by hydrogen bonding. Very hard to remove, it exists as long as there is a surface. The binding force is really strong.
\textbf{Lower density}, \textbf{higher viscosity} compared to bulk water. Very critical part!
Water of Hydration:\\
Chemical bound water with the particles, only removable with the expenditure of thermal energy (ex. drying).\\
dewatering -> 20 percent
thermal drying -> 95 percent
Rheology:\\
Science of flow and deformation of fluids.\\
Commonly measured in terms of \textbf{viscosity}.\\
Shear force applied, a velocity distribution occurs.
\[\tau = \mu * \frac{du}{dy}\]
$\tau$ is shear stress,\\
$\mu$ is viscosity,\\
du/dy is the shear rate.\\
Viscosity is dependent on temperature.\\
sludge $\rightarrow$ mostly not Newtonian but if it is diluted it is Newtonian.
It is not about thickness, thick fluids can be Newtonian.

 \begin{enumerate}
     \item Plastic fluids:\\
     There is a intercept (yield stress, that should be overcome to make the fluid flow) in the formula, but it looks like Newtonian.
     \item Pseudoplastic fluids:\\
     \textbf{Most} wastewater sludges are neither Newtonian nor plastic non-Newtonian.\\
    K = fluid consistency index, analogous to viscosity,\\
    n is flow behavior index,\\
    n < 1 for this sludge flow.
    \item Dilatant fluids:\\
    We don't see that in sludge.\\
    same as pseudoplastic fluid, but
    n > 1.
 \end{enumerate}
Sludge is believed to be a mix of pseudo-plastic and plastic, which means with a intercept decreasing curve.\\
Viscosity of sludge changes over time. It is a thixotropic fluids, since we break flocks while adding shear stress, a hysteresis (it does not follow the same curve) loop occurs.\\
$\mu = \mu_0(1+2.5\phi)$ Einstein's Equation of Viscosity\\
$\phi$ is the volume fraction of particles.\\
The particles in the medium increases the viscosity, follows 10\% volume fraction but this is for molecules that don't interact with each other.\\
4 parameters: shear stress, time of measurement, temperature, solids concentration.\\
Specific Resistance to Filtration as a Dewaterability Indicator:\\
Belt filter, vacuum filter were mostly based on filtration.
Now, it is based on centrifugation.
Darcy's law is based. % add the darcy's law 
Carman adapted Darcy's law for filtration.\\ % add the Carman
The resistance is contributed by the filter medium and the cake.
Coakley and Jones adapted Carman's theory.\\
They were noisy, electricity use too much, too much space.\\
Resistance for sludge is better term so R = 1/K.\\
Specific resistance is  time / volume vs. volume.\\
Theta divided by V, certain slope and intercept, which is filter media resistance.\\ % put the formulas
Specific resistance = Buchner funnel apparatus.\\
Definitely gives you dewaterability with filtration based system
inorganic/non compressible minerals, vacuum filtration is used.\\
w = w/v (almost like density)\\% cake deposited per unit volume % derive the formula of w
A new test is developed.\\
CST (capillary suction time) apparatus.
Gives an idea for 1 cm movement of water, developed in England
national lab, comparing sludge, best conditioners
function of:
\begin{enumerate}
    \item CST filter properties
    \item Instrument properties
    \item Sludge related properties
\end{enumerate}
It is an empirical model by Vesilind. % add the formula
\begin{itemize}
    \item sensors, paper thickness, collar size, paper absorbency
    \item sludge sample, Fluid viscosity, solids concentration
\end{itemize}
SRF $\rightarrow$ higher, harder to filter, resistance\\
x (cy) $\rightarrow$ higher, easier to filter, filterability\\ % change the shape of that X
they both independent, both are good, SRF is easier to measure but correlations might help.\\
\textbf{Chemical characteristics:}\\
Presence of organic materials make it complicated.\\
Important for the final destination of sludge.\\
Dry sludge has a fuel value up to 5500 cal/g dry volatile solids (not a bad number), Coal has a fuel value of 7700 cal/g, very good coal.\\
Unfortunately, sludge neither dry not it contains only the organics 550 cal/g.\\
Combustion of sludge usually requires the use of auxiliary fuel, treatment decreases the calorific value.\\
66\% Turkish lignite around 1000 and 2000.\\
For industrial sludges, the calorific values varies. Most of them are higher than the lignite in Turkey.\\
Municipal sludge average is \textbf{3580 kcal/kg}. And they don't show much variations.\\
Fertilizer value of Sludge is another important parameter.\\
Nitrogen, phosphorus and potassium (3 * 8\% rule) is expected.\\
Micro-nutrients can be provided.\\
Sludge contains less amounts, sludge is not full scale. However, sludge is a perfect soil conditioner.\\%what does that mean?
Other problem is heavy metals, chlorinated hydrocarbons, and pathogens.\\
Electrical charge of sludge is also considered.\\
A negative net charge in surface for sludge (-20 to -30 milivolts). Since it is a large negative number, dewatering and flocculation are affected.\\
Metal concentrations are concerning in sludge as well.\\
Not allowing the metals get into sludge is the best solution to reduce them in the sludge.\\
\textbf{Biological Characteristics:}\\
Microbial quality of sludge is crucial due to:
\begin{itemize}
    \item different groups of organism in the sludge for the balance
    \item presence of pathogens in sludge
\end{itemize}
Dominating? Based on operational conditions.\\
Pathogens?? Activated sludge contains less pathogens than raw primary sludge but still contains. We usually report them number per g dry solids.\\
Shell type ones are resistant.\\
Ascariasis $\rightarrow$ not common but a concern in Africa, eggs are infecting the body.
\textit{Surface Polymers}\\
Polysaccharides outside the bacteria: able to attach, flocs formed, they are resistant.\\
\textbf{Current issues:}
\begin{itemize}
    \item micropolutants in the sludge
    \item emerging contaminants
    \item hormones
    \item detergents
    \item microplastics
    \item disinfectants
    \item plasticizers fumigants
    \item pesticides
    \item nanomaterials
\end{itemize}
The concentration really low, but they stay.\\
Why?\\
Negative impact on aquatic organisms, ex. feminization of male fish.\\
Bio-accumulate in fatty tissues.\\
Toxic or Carcinogen.\\
Endocrine disrupting ones.\\
Source control is the most affecting.\\
Use of more effective technologies might help in the future.
\date{\begin{flushright}March 31, 2023\end{flushright}}
\section{Sludge Stabilization}
stabilization defines the operations and processes carried out on sludge to minimize its damage to the environment when sludge is disposed of.
i\begin{itemize}
    \item Reduction of Volatile Suspended solids (VSS)\\
    it indicates microbial content activity, reduction helps with stability for reducing it.\\
    70-80\% is VSS in sludge.\\
    Standard is based on this.
    \item Odor Reduction\\
    Various sources causes, hydrogen sulfer is the most ipmortant constitient.
    sulfurous and nitrogenous compounds smell.\\
    H2S -> detection limit is 0.040 and olfactory threshold concentration 0.4
    anaerobic sulfur cycle originates hydrogen sulfide gas. low concentaartions, terrible smell, in high concentations H2S has a rather pleasant smell, but it is lethal.
    The olfactory threshold limit for H2S is 1.3e-3 but % add here the rest
    \item Pathogenic Microorganisms\\
    Pathogens are always in the sludge. % add here the log reduction and percent reduction
    \item Toxicity\\
    two questions: toxicity to what? and what is the  ultimate destination?\\
    generally wastewaater sludge consists of high quliaities of heavy metals, these are toxic to most organisms; these cannot be stabilized, since they are not biodegredable, they might be reduced to a certain level, prevention is the better option.\\
    intended use plays an important role.\\
    land application % add the reasons\\
    the prevention regulations helped a lot with Ankara Sludge content.
    \item Change in Oxygen Uptake Rate
    \item Gas Production % add details
\end{itemize}
Method of Stabilization
\begin{itemize}
    \item Biological
    \item Chemical
    \item Physical
\end{itemize}
Biological is the most common, which is the most effective.,
Conventional digester  % bold
is a large tanl with either or floatinbg or a fixed cover. Sludge is pumped into the middle and stabilized solids are removed from the bottom. Since the sludge is not mixed, the contents of digester separate into layers of scum, supernatant and active and stabilized solids. The disadvantage they require quite long detention times.
High rate digester % bold
Mixed either mechanically or by circulating the compressed digester gas.
Two fazed digestion:
Heated then primary digester (mixed) then settling in secondary digester
Supernatant sent back to the starting of primary treatment
This is called, two stage high rate anaerobic digester.
The gas might be pumped back into the system for mixing.
Perperllers can be used for the mixing aas mechanical but it is superficial.
The nozzlers can also be used.
Cylinder ones are called "pancake"
Egg shaped ones to prevent dead zones in the digester.
Steps of Digestion:
Step 1. Solubilization Step in which the complex organics are degreadted by the extracellualar enzymes into slouble organics and simple sugars. Slow degredation, enzym activity is really important, a real limiting step. % bold
Step 2. Acid Formation Step in wihch  the soluble organics are degreaded by a group of microorganisms known as acid formers.
Acid formers are not the problem for the sludge, they are in the sludge and not affected by their environment easily. Therefore, this is not a limiting step.
Step 3. Methane Formation Step in which the methane forming bacteria which are strictly anaerobic take the organics acids and convert them to methane and carbon dioxide. Another difficult step, very limiting step. pH is very sensitive. The environment is really important, hence alkalinity is an important component in anaerobic treatment. They are not really rate limiting but they are the most crucial point of the digestion. % ask
%%% there are formulas that can be inserted here esp. Buswell Equation
Pretreatment to Increase the efficiency
Use of chemicals like NaOH, ozone, physical techiniques such aas blending or ultrasonication.

\date{\begin{flushright}April 7, 2023\end{flushright}}
stabilization part
anaerobic composting is not possible.

\date{\begin{flushright}April 14, 2023\end{flushright}}
Transportation of sludge is an issue, pumping 15 km is same as dewatering, why do we do that because it might not be available for the treatment system.
Liquid 1 10
Dewatered Sludge 10 35
DDried 35 over
Low concentrations 5\% approximately water
ex clarifiers 1.2 1.5\%
turbelent with 0.3 m/s
It is important it is turbulent happening
- Sludge solids
- Sludge rheology
- volatilke solids
- temperature
Newwtonion fluids -> pressure drop is proportional to the velocity and visscocity, for non, it is not proportional.
loses = exit loses + minor losses + frinction losess
Friction losses is the most important, 3 equations
different expertise, still dilute sludge
Manning equation, be careful about looking at the empirical equaation, stick to the units
R, cross sectional area / wtter premeter, S is slodpe of the energy grade line
Open channel flows
such as sewers
Hazen Williams
another empirical one
here as roungness increases, velocity degreses in this formula
commonly used for full pipe
Darcy-Weisback Equation
universal
f is function of R
the complication is mu, dependent of shear rate and time, thixotropic material
the formula for reynolds equation is modified:
both applicable for non and newtonian
f = 64/R for laminar
the modified version takes the form of normal version when you put the right numbers
curve fitting into K and n
redoe this everytime with new sludge
Alternatively a solid correction factor can be applied after the headloss calculations for water according to solids and velocity
steel
cast iron
concrete
pvc
150 mm 200mm
cleaning is a must because of clogging
short distances up to 20\%
solids concepts up to 8\%
kinetic pumps and positive displacement such aas plugnger pumps diagramp pumps and rotery pumps
Kinetic (centrifugal) impoler rotaroyu -> pulls the sludge and pushes
clugging problem for sludge, continuous mode so it is good for dilute ones
postiive displacement -> pull and push (pulsating motion) theory such as piston or diafram
Rotary screw pumps, 9-12 m upwards
dewatered sludge pumping?
15 to 40\% we talk about pudding
mechanical conveyors, screw augers or best is truck transportation
hydrolically driven reciprocating piston pumps* for the high solid concentration used for 20 years by municipals
dried sludge conveying are done with mechanical conveyors.

\date{\begin{flushright}April 28, 2023\end{flushright}}
sludge thickner
very important, also changes the concentration, first method to reduce volume
const effectiveness of the treatment
even solid concentration increases only 1\% is not high, volume reduction is the crucial part.
digesters are costly operations. thickneres are before them. requires less volume
thickener removes the water
thickner vs dewtering
thickner -> lower than 15 solid concentration
dewatering -> higher than 15 solid concentration
for instance: centrifugion
thickened still acts like a water, dewaater usually solid
gravity thickeners
floatition thickeners
centrifugal thickener
gravity belk thic
rotatory drum
gravitty thicknere -> most likely a settling tank, bottom part collects the solid
circulars
3 to 4 m depth 21 to 24 m, detention time is increased, long detention time anoxic conditions, gas production.
slope is great for minimum solid retention, reduces raking transport problems
gravity settler and thickner,
thickner is longer retention time, 24 hours typically, clarification and thickenening achiveeved.
the longer we wait, floating sludge occurs, because of nitrification conditions.
can be based on experiences, also lab data
some parameters are given, but there are example of lab designs
solid flux: kg solids/h/m2 % they have already values
tabulated activated sludge
tricking filter humus
raw primary sludge
raw primary and WAS
pure o2 activated sludge
A = Q/Flux
it gives area
inf solid concentration, thickening time, thickened solid conc, dry solids loading
lab setup ddesgin:
lab settling data.
typically 1000 mL
sludge starts to settle with time and it is plotted
height vs time graph, and the velocity is tangent line to that ZSV
cleared water, hindered zone, transation zone, compaction
compaction occurs, where the magic happens
more solid consentration which results in ZSV vs solid concentration
hinderence is the cause of parabolic inverse relation
flux = velocity * solid concentration
flux vs solid concentration can be retrieved as well (more like log gaussian distrribution)
Feed -> Effluent
two ways of solid movement:
Under the influence of their settling velocity
due to continuous removal of sludge at the bottom of underflow
settling velocity -> batch settling
second mechanism, underflow flux -> independent of the solids settling in the thickener
Gu = uCi
underflow flux -> straight line due to u being the m
total flux (Gi) = uCi + ViCi
it does not show batch flux, they are both added, total flux
Co is entering Cu ultimate goal
CL value, limiting flux, which means GL, it is the design limit, minimum solid loading (GL) we calclate for the maxixmmum area reqiured.
Total flux curve gives it. Simpler method, solids flux curve from Cu tangent line to graph gives Cl on x and GL on Y value. Both cases, GL is the same
A = Q0C0/GL
floatation sludge
float them up to surface using bubbles
disolved air floatation is most common used.
fine gas bubbles to the liquid phase, solid particles, skemmer collects them
lighter sludges work better, WAS aerobically contact activated, extended aeration, not used for primary  or trickling sludge (these are ssettled by gravity better in terms of economically)
Advantages:
small and light particles
higher solid loading can be treated with DAF
less space
Disadvantages
requires pilot scale tests
daf systems are highly mechanized
operation and maintenance costs are significant
collision -> attacment -> detachment
A/S (mL air/mg solids) ratio is important, characteristics of sludge, distirbution of air, air dissolvation efficiency, it is important because the blanket can be escaped.
Based on: surface loading rate, detention time, solids loading rate.
again solid capture rate, solids recovery rate, in terms of percentage, how well you capture the solids, thel lower of the solids in the effluent, higher this number
Air to solids ration, surface area, detention time, the recycle rate can be achived.
formulas for the aeration
the conclusion: floation less expensive for activated sludge, 1\% solids will result in 4-6\% esp when polymers are used
gravity belt thickeners
1980s
belt moving, sludge can sit
gravity and coagulation and flocc, which is conditioning
while on the belt some water has been lost, by plows
sludges that are good: aerobically and anaerobicall digested solids, alum and lime solids, primary solids, WAS, blended sludge between 0.4\% to 8\%
how well: solid capture rate, solid concentration up to 6\%
primary sludge can be used for good good solid concentration
rotary drum thicknere:
we use again polymer
drain water from the rotary drums using different tools in the thing
solids recovery -> only the rest of the solids escaping
WAS, anaerobically and aerobically digested sludge and some industraial ssludge
very good solid capture
smaller space requirements relatively low capital cost low power consumption
critical polymer, good odor keeper
small to medium wastewater treamtment
Extra:
SVI = mL sludge * 1000 / MLSS (mg/L)
not dependent on solid concentration, may result in bad cases
a not settled 1000 mL sludge with 10000 mg/L MLSS content gives 100, which is good in terms of numbers but in reality it is bad.
40 120 well settled
120 nulking
150 severely bulking
bulking -> hard to settle with gravity
1. viscous bulking
2. filamentous bulking
1. viscous bulking: so much going on, extracellular polymers which is caused by excessive carbohydrate loading to the system C/N ratio of 40
2. Filamentous microorganism, Low DO, Low F/M, Nutrient deficiency, Toxic subbstances
why problem?
low return activated sludge
WAS is low
in severe cases sludge blamnked öy overflow
how to solve filamentous bulking:
11. you have to know the exact problem in the system
manipulation of return activated sludge flow, to regulate both DO and the organic concentration in the tank
2. addition of chemicals to imvporve settling
3 addition of toxicant sto selective lill the flamentous bactreria
last two are temporary fixes

\date{\begin{flushright}May 5, 2023\end{flushright}}
conditioning: chemical physical thermal process to improve the efficeincy of thicknening or dewatering operations
increased thickening and dewatering
stabilization
odor control
does not reduce water content!
alters the phyical properties.
work better with dewatering and thickening economically
chemical: inoranic and organic polymers, conventional
thermal: heating andw freezing
physical: ash from thermical powerss, elutriation; very old methods
factors:
source of sludge; secondary needs more condition
solids concentration
particle size and distribution; the surface area/volume increases, needs more conditioning
pH and alkalinity
surface charge and degree of hydration
other physical factors
chemical:
inorganic: alum, iron, lime
orgainc: ployelectrolystes; mostly used, more expensive
two mechanism: charhge neutralization, bridging the particles
most particles icluding bacterial negatively charge
the energy barried sshouldw be slipped in order to come together
hydroxy mechanism also workds for the neutralization via their formation until the solubility limit exceeded.
organic polytmers do bridging better, built by monomers
functional groups give it a charge. mostly cationic ones. logically anionics are not suitable but H+ ion introduced, they can be used.
with high molecular weight, bridging and neutralisation done by them, they form strong forms.
catoinic polymors -> more expensive,
pwerder pellet, liquid -> requires to be dilute 0.01\%, they should strech out
easy oepration, safe handing, better thermal characteristics
mass increased significantly, which is a disadvantage
CST -> used because of water relaease
Jar test -> because settling
rotational viscometer -> torque vs time
thermal conditioning - heat treatment:
heating causes the breakdown
porteus process -> dissolve microbial content, system -> this bring a high COD and nutrients back to the system start, so you need to design for that.
freeze/thraw -> freezing sludge later releases water nicely, also reduces the pathogens, disadvantages -> cold cilmate is required.
elutration -> ions reduced, lwoers the alkalinity, reduces the chemical need for conditioning, washwater is added.



\date{\begin{flushright}May 12, 2023\end{flushright}}
Sludge dewatering -> removal of water
behave solids than a liquid
non-mechanical and mechanical
objectives:
high oslids concentration in the filter cake
a lean filtrate/centrate
an acceptable yield (mass of solids collected/are of dewatering system time)
machine related
operational (sludge) related
calculate solids recovery -> how much of the incoming solids are captured in the cake, how little we are releasing to outflow
0 -> inflow
K -> cake
F -> outflow (filtrate/centrate)
\%R = mass of dry solids as cake / mass of dry feed solids
1. Drying beds
depth 15-30 cm, several weeks to months to dry
gravels
sludge collection:
manual -> better, more careful
forklifts -> we lose sand
water drains through sand until sand is clogged
drainage
25\% if not conditioned, 75\% in good condition removal of water
evaporation
sludge, and climate
disadvantage: climate
disadvantage: high land requirement
advantage: small plants are good, low electricity, low requirement for operator attention
design: soılids loading rates kg/m2-y
best way to design: pilot studies
covered area requires less area.
2. lagoons
no underdrain system, very large volumes required, 1 m sludge, odor intenstive process.
3. pressure filters
filter press, not right name, not pressing
pressurized so that water is removed, cake is removed from the surface
it is a cyclic oepration, very disadvantageous in large treatment systems, generally effective
another disadvantage, cleaning aspect since not aesthetically acceptable
filter presses 50:50, 35 to 47\%
very good
solids content, chemical conditioning dose, total scycle time, solids capture, desired yield
design paramters:
coake ssolids conceytration, yield rate, recovery fraction
cycle time increased -> high solids concentration in the cake
95\% recovery,
sizing the dewatering equipment depends on the suppliers
4. belt filter presses
very much loved in turkey, low capital system, versatile, low-speed
in S zone, shear and direction changes, 20-25\% achived.
5. centrifuges
high force 500 - 3000 G force so that solids then sludge is scraped from the walls
solid bowl crntrifuge
feed flowrate, rotational speed, chemical use
oepration: ssolid bowl bullet shaped casing, which is outside
scroll and a screw conveyor that is placed in the middle of the centrifuge
objectives:
dry cake, clear centrate, reasonable throughput
clarification of the liquid
successful movement of the cake
general:
centrigufe: relative less space,
belt filter: easy to operate
presure filter: high cake solids concetration
drying nbeds low capital costs
centrifuge: relaitvely high costs, direct energy
belt filter: very sensitive feed sludge, sentite to plymer type
oressure filter: batch oepration, high capital and labor costs
drying beds and lagoons:large area requirement, stabilizaed sludge requirement, climate effect
\date{\begin{flushright}May 26, 2023\end{flushright}}
7/8

\end{document}
